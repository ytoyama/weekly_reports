\documentclass[aspectratio=43,unicode,10pt]{beamer}
\usetheme{ttipresentation}

\usepackage{luatexja}
\usepackage{luatexja-fontspec}
\usepackage{graphicx}
\usepackage{multicol}

\setmainjfont{ipagp.otf}
\beamertemplatenavigationsymbolsempty

\newcommand{\itemtitle}[1]{\textbf{#1}\\}
\newcommand{\fire}[1]{\textcolor{red}{\textbf{#1}}}
%\newcommand{\freeze}[1]{\textcolor{blue}{\textbf{#1}}}
\newcommand{\then}{\textcolor{ttiblue}{\textbf{⇒}}\hspace{1ex}}
\newcommand{\good}{\textcolor{orange}{\textbf{◎}}\hspace{1ex}}
\newcommand{\arrow}{\textcolor{ttiblue}{\textbf{→}}\hspace{1ex}}
\newcommand{\mb}[1]{\mathbf{#1}}


\title{今週の進捗}
\institute{知能数理研究室}
\author{12056 外山 洋太}
\date{\today}



\begin{document}

\begin{frame}
\titlepage
\end{frame}

\begin{frame}{新しいモデルの実装}
  \begin{block}{char2word2sent2doc by someone, NAACL 2016}
    \begin{itemize}
      \item 三輪先生から貰った論文が元
      \item 元論文はword2sent2docだった
        \begin{itemize}
          \item 単語のembeddingsから文、文書のembeddingsを順に
                Bi-directional atttentioned GRU RNNで作成
        \end{itemize}
      \item 同じembedding生成の方法
        \begin{itemize}
          \item Attention付き
          \item Bi-directional
        \end{itemize}
      \item パラメータは主に文字のembeddingsとGRU、分類のための全結合層
      \item char2word, word2sent, sent2docのRNNは全て同じ実装
        \begin{itemize}
          \item パラメータは別で持つ
          \item 実験ではハイパーパラメータはほとんど一緒
        \end{itemize}
    \end{itemize}
  \end{block}
\end{frame}

\begin{frame}{新しいモデルの実装}
  \begin{block}{ハイパーパラメータ}
    \begin{itemize}
      \item embeddingサイズ(一方向)
        \begin{itemize}
          \item 文字:32
          \item 単語:32
          \item 文:32
          \item 文書:32
        \end{itemize}
      \item 隠れ層の数:1
      \item 隠れ層ニューロン数:64
      \item (出力層ニューロン数:1)
      \item ドロップアウト率:0.6
      \item L2正則化係数:1e-6
      \item context vectorサイズ:32
    \end{itemize}
  \end{block}
\end{frame}

\begin{frame}{新しいモデルの実装}
  \begin{block}{その他実験設定}
    \begin{itemize}
      \item 学習回数:1024
      \item バッチサイズ:1024 (今後減らします。)
      \item 訓練データサイズ:25'000
      \item 評価データサイズ:25'000
        \begin{itemize}
          \item 各々positive, negativeが12'500ずつ
        \end{itemize}
    \end{itemize}
  \end{block}
\end{frame}

\begin{frame}{新しいモデルの実装}
  \begin{block}{実験結果}
    実験が終わっていません
  \end{block}
\end{frame}

\begin{frame}{TensorFlowとDockerとCUDA}
  \begin{block}{TensorFlowの現在}
    \begin{itemize}
      \item Linuxの公式サポートはUbuntu 14.04 LTSのみ
      \item CPU版はFedoraでも動く
      \item GPU版はUbuntuとFedoraでディレクトリ構造が少し違うのでそのままでは
            動かない \\
            \then 他の方法
        \begin{itemize}
          \item \good 公式のDockerイメージ
          \item chroot?
        \end{itemize}
    \end{itemize}
  \end{block}
\end{frame}

\begin{frame}{TensorFlowとDockerとCUDA}
  \begin{block}{公式のDockerイメージ}
    \begin{itemize}
      \item ホスト側Linuxの公式サポートはUbuntu 14.04 LTSのみ
      \item Githubのリンク (https://github.com/tensorflow/tensorflow/tree/master/tensorflow/tools/docker)
      \item すごく速い(TITAN X x 3で100 iterations / hour以上)
        \begin{itemize}
          \item CPU x 16 (hyperthreading)で半日100 iterationsくらい
        \end{itemize}
    \end{itemize}
  \end{block}
\end{frame}

\end{document}
